\chapter*{Conclusion}

\tab The rise of Space and Geospatial Technology had a deep impact on all layers of society. By combining a highly technological space infrastructure (earth observation, positioning and communication) with new technologies for data utilization (AI, IoT, etc.), the contribution of SGT for the economy is already visible but should be further promoted (\ref{wi_sgt}). 

\vspace{0.4 cm}

More specifically, SGT could participate in the realization of the 2025 ASEAN vision of increased resiliency and connectivity. In 2016, the \textit{Master Plan on ASEAN Connectivity 2025} itself reaffirmed the importance of space technologies and data sharing for regional economic development (\ref{enhanced}). As explained previously in this report, SGT can play a prominent role in the global optimization of ASEAN production system (\ref{optimization}) and the transformation of the economy with the third unbundling (\ref{unbundling_p}). SGT will also help ASEAN to develop an ambitious vision towards the status of Data-Driven Innovation Economy 2.0.

\vspace{0.4 cm}

It is therefore primordial for ASEAN member countries to develop a common vision and common policies towards the establishment of an efficient physical infrastructure for data collection in ASEAN (\ref{infrastructure}). It will also be necessary to create a regional data policy for data utilization and data format standardization (\ref{data_util}).

\vspace{0.4 cm}

Finally, several ambitious flagship projects should be implemented for the increase of both land (\ref{land}) and sea connectivity (\ref{sea}) and for the continuous development of human resources with adaptive systems (\ref{flagship}). It will support the promotion, maintaining and enhancement of the feeling of "We" and deepening the sense of ASEAN identity.




