\chapter*{Executive Summary}

Optimization of production processes is the most prioritized issues in the industrial factories. Such optimization is quite feasible because it is easy to acquire monitoring data  of environment and boundary conditions which are connected to productions. Besides, it is easy to project impacts of process changes. However, it is a challenge to extend the optimization beyond the factory processes, such as delivery to consumers and retail stock. It is  because factories do not have accesses to information on trends and amount of purchase and latency and incidents during cargo stowage and transport routes during delivery processes. These are actually uncertainties in optimization of the whole processes. For example, retails have to stock more than observed purchases for risks of unexpected latency of production and delivery while productions have to estimate longer periods for delivery than feasible periods due to uncertainties of incident occurrences.

Based on this, constructing large scale production systems by connecting variety of sectors like factory, delivery, and retail yields uncertainties between the connections and results in degradation of production efficiency.

Space and Geospatial Technology (SGT) had been developed as a intelligence technology. Space infrastructure, such as earth observation, positioning and tracking, and communications, helps more effective decision making to reduce uncertainty and risks by continuous monitoring and visualization of situational information of the real world, such as people mobility and activity, vehicle traffics, meteorology and oceanology,  and disaster.

Potentials of space infrastructure are rapidly growing along with the successes of small-scale and low-cost satellites. In addition, data infrastructure for big data and artificial intelligent technologies enables rapid integration and analysis of real world dynamics by variable data resources of earth observation and positioning and tracking from the space  infrastructure. SGTs support decision makers to swiftly monitor and accurately predict the situation and changes of people and companies. SGTs therefore facilitate more safety and less risk in decision making and activities by providing the dynamic situational information of general industry and people's lives.

For instance of heavy rain disasters, rapid areal monitoring flooded roads and traffic jams helps delivery services to response to reduce impact of service latency in the disaster situation by changing transports and routes. The reduction of uncertainty in delivery facilitate planning of production and retails with shorter production periods and less stocks.
Besides, SGT helps to record disaster responses as data which is useful for AI-based analysis to improve responses to future disasters.

This kind of SGT-based integration supports construction of large-scale production-delivery systems integrating many enterprises over various fields by reducing uncertainties caused by external factors. As the result, more various products and services are provided in more efficient manners.

In the ASEAN region, where many and various organizations and enterprises expand over large geographic extent and population, it is very important to proceed "unbundling" and "rebundling", with which communities and companies recognize their own roles and connect them over the ASEAN region, for the sustainable development. SGT should be an indispensable infrastructure of information sharing needed for the "unbundling" and "rebundling"

While the ASEAN prioritize to secure people's safety and stabilize living levels, SGT is also contributes to better resilience of the ASEAN region, which is the most populated and disaster-prone areas in the world, by providing proper dissemination and navigation against the disaster risks in their lives.

To proceed SGT-based integration in the ASEAN region, it is needed to promote public-private-partnership (PPP) of space infrastructure development and supporting ground infrastructure. Another important issue is establishing data policies facilitating trans-broader data transfer and utilization under proper management mechanism. This report provides recommendations about strategies and frameworks of ASEAN's data policy and space infrastructure development.
