\chapter{Summary of field visits and interviews}

\tab This appendix presents the list of field visits and interviews carried out for this project. They aimed at gathering the concerns, needs and experience from regional practitioners, both from ASEAN governments and regional institutions.

\vspace{0.4 cm}

All visits and interviews were carried out by Prof. Hiroyuki Miyazaki and Mr. Quentin Verspieren.

\section{Nay Pyi Taw, Myanmar -- October 25, 2016}

\begin{table}[H]
   %\caption{ }
   \centering
   \begin{tabular}{| c | p{12 cm} |}
   \hline
    Place & Emergency Operations Center (EOC), Relief and Resettlement Department (RRD), Ministry of Social Welfare, Relief and Resettlement (MSWRR), Nay Pyi Taw, Myanmar \\ \hline
    Date (Time) & October 25, 2016 (10:00--12:30) \\ \hline
    Counterpart & Ms Thiri MAUNG, Deputy Director of the EOC \\ \hline   
   \end{tabular}
\end{table}

Main elements of the discussion:

\begin{itemize}

\item Role and responsibilities of the EOC.

\item Future version of the \textit{Myanmar Action Plan for Disaster Risk Reduction}.

\item Importance of improving emergency information dissemination.

\item Consequences of the Nargis Cyclone.

\item Overview of Myanmar's international collaborations for disaster reduction.

\item Structural problems of disaster management in Myanmar

\end{itemize}



\section{Vientiane, Lao P.D.R. -- January 30, 2017}


\subsection{Disaster preparedness and response division, MONRE}

\begin{table}[H]
   %\caption{ }
   \centering
   \begin{tabular}{| c | p{12 cm} |}
   \hline
    Place & Disaster Preparedness and Response Division (DPRD), Department of Disaster Management and Climate Change (DDMCC), Ministry of Natural Resources and Environment (MONRE), Vientiane, Lao P.D.R. \\ \hline
    Date (Time) & January 30, 2017 (9:45--11:25) \\ \hline
    Counterparts & \begin{itemize} \item Ms Sonephet PHOSALATH, director of the DPRD \item Mr. Xailee XAYAXANG, technical officer at the DPRD  \end{itemize} \\ \hline   
   \end{tabular}
\end{table}

Main elements of the discussion:

\begin{itemize}

\item Organization of disaster management in Lao P.D.R.

\item Governmental efforts and inter-ministerial collaborations for disaster management.

\item Access to geospatial data from international partners (Japan and Sentinel Asia in particular).

\end{itemize}


\subsection{Natural Resources and Environment Institute, MONRE}

\begin{table}[H]
   %\caption{ }
   \centering
   \begin{tabular}{| c | p{12 cm} |}
   \hline
    Place & Office of the Deputy Director General, Natural Resources and Environment Institute (NREI), Ministry of Natural Resources and Environment (MONRE), Vientiane, Lao P.D.R. \\ \hline
    Date (Time) & January 30, 2017 (11:25--12:30) \\ \hline
    Counterparts & \begin{itemize} \item Mr. Phetsamone DALALOM, Deputy Director General of the NREI \item Ms Sonephet PHOSALATH, director of the DPRD \item Mr. Xailee XAYAXANG, technical officer at the DPRD \item 3 technical officers of the Remote Sensing Center (RSC), NREI \end{itemize} \\ \hline   
   \end{tabular}
\end{table}

Main elements of the discussion:

\begin{itemize}

\item Presentation of the Remote Sensing Center's activities and current capabilities.

\item Interactions between the RSC and Sentinel Asia.

\end{itemize}


\subsection{Department of Meteorology and Hydrology, MONRE}

\begin{table}[H]
   %\caption{ }
   \centering
   \begin{tabular}{| c | p{12 cm} |}
   \hline
    Place & Department of Meteorology and Hydrology (DMH), Ministry of Natural Resources and Environment (MONRE), Vientiane, Lao P.D.R. \\ \hline
    Date (Time) & January 30, 2017 (13:50--14:30) \\ \hline
    Counterpart & Mr. Bounteum SYSOUPHANTHAVONG, Head of the Forecasting Division \\ \hline   
   \end{tabular}
\end{table}

Main elements of the discussion:

\begin{itemize}

\item Overview of the current capabilities of the DMH concerning meteorological and hydrological stations.

\item Radar facilities of the DMH.

\item Access to meteorological information from the People's Republic of China, Japan and South Korea.

\item Responsibilities of the DMH for disaster management.

\end{itemize}


\section{Hanoi, Vietnam -- March 7-9, 2017}

\tab This field work consisted in participating in a three-day summit on the Sentinel Asia initiative. It consisted in several events:

\begin{enumerate}

\item Celebrations for the 10th anniversary of Sentinel Asia (March 7)

	\begin{itemize}
	\item Past achievements of Sentinel Asia: good practices.
	\item Long-term strategy for Sentinel Asia.
	\end{itemize}

\item Data-Analyzing Nodes (DAN) and Data-Providing Nodes (DPN) meeting consisting in reports from all DAN/DPN organizations (March 7)

\item 4th Joint Project Team Meeting (March 8-9)

	\begin{itemize}
	\item Presentations and feed-backs from most member countries and regional organizations (e.g. ADRC and ADB).
	\item Recommendations for the future of Sentinel Asia
	\end{itemize}

\end{enumerate}


\section{Jakarta, Indonesia -- May 30-31, 2017}

For this field work, Prof. Miyazaki and Mr. Verspieren were joined by Mr. Koji Suzuki, Executive Director of the National Research Institute for Earth Science and Disaster Resilience (NIED), Japan and Mr. Naoki Yamaguchi from Asia Air Survey Co., Ltd., Japan.

\subsection{ASEAN Secretariat}

\begin{table}[H]
   %\caption{ }
   \centering
   \begin{tabular}{| c | p{12 cm} |}
   \hline
    Place & ASEAN Secretariat, Jakarta, Indonesia \\ \hline
    Date (Time) & May 30, 2017 (10:00--11:00) \\ \hline
    Counterparts & \begin{itemize} \item Ms Malyn TUMONONG, Assistant Director/Head of the Disaster Management and Humanitarian Assistance Division (DMHA) \item Ms Pimvadee KEAOKIRIYA and Mr. Chandra Satriadi PUTRA, Senior Officers, DMHA \item Mr. Tika Savitri PANDANWANGI, Technical Officer, DMHA \item Mr. Dimas ADEKHRISNA, Technical Officer, Science and Technology Division \end{itemize} \\ \hline   
   \end{tabular}
\end{table}

Main elements of the discussion:

\begin{itemize}

\item Presentation of our project.

\item Main role of the DMHA division with regards to other institutions in the region.

\item Opportunities of presenting our project to the ACDM or COST-SCOSA.

\end{itemize}


\subsection{BNPB - Indonesian National Disaster Management Agency}

\begin{table}[H]
   %\caption{ }
   \centering
   \begin{tabular}{| c | p{12 cm} |}
   \hline
    Place & BNPB Headquarters, Jakarta, Indonesia \\ \hline
    Date (Time) & May 31, 2017 (10:00--11:00) \\ \hline
    Counterparts & \begin{itemize} \item Dr. Munadjat BAMBANG and Dr. Fuadi DARWIS, BNPB Advisory Board Members \item Ms SUSILASTUTI, Staff of the Preparedness Directorate \item Mr. BAMBANG, Senior Officer, Early Warning Sub-Directorate \end{itemize} \\ \hline   
   \end{tabular}
\end{table}

Main elements of the discussion:

\begin{itemize}

\item Presentation of our project.

\item Main challenges facing Indonesia.

\item Role and responsibility of the BNPB.

\item Importance of common regional standards for data sharing.

\end{itemize}


\subsection{AHA Centre}

\begin{table}[H]
   %\caption{ }
   \centering
   \begin{tabular}{| c | p{12 cm} |}
   \hline
    Place & AHA Centre, Jakarta, Indonesia \\ \hline
    Date (Time) & May 31, 2017 (11:15--12:00) \\ \hline
    Counterpart & Mr. Suryo ANDRI, Communications Officer \\ \hline   
   \end{tabular}
\end{table}

Main elements of the discussion:

\begin{itemize}

\item Role and responsibilities of the AHA Centre.

\item Functioning of the AHA Centre during emergency operations.

\end{itemize}



\section{Bangkok, Thailand -- June 1, 2017}

For this field work, Prof. Miyazaki and Mr. Verspieren were joined by Mr. Koji Suzuki, Executive Director of the National Research Institute for Earth Science and Disaster Resilience (NIED), Japan and Mr. Naoki Yamaguchi from Asia Air Survey Co., Ltd., Japan.

\subsection{United Nations Economic and Social Commission for Asia and the Pacific}

\begin{table}[H]
   %\caption{ }
   \centering
   \begin{tabular}{| c | p{12 cm} |}
   \hline
    Place & Information and Communications Technology and Disaster Risk Reduction Division, United Nations Economic and Social Commission for Asia and the Pacific (UNESCAP), Bangkok, Thailand \\ \hline
    Date (Time) & June 1, 2017 (11:00--12:00) \\ \hline
    Counterparts & \begin{itemize} \item Dr. Tae Hyung KIM, Economic Affairs Officer \item Ms. Youjin CHOE, Consultant \end{itemize} \\ \hline   
   \end{tabular}
\end{table}

Main elements of the discussion:

\begin{itemize}

\item Presentation of our project.

\item Role and responsibilities of the UNESCAP.

\item Potential of remote sensing data for the Asia-Pacific region.

\end{itemize}


\subsection{Geo-Informatics and Space Technology Development Agency}

\begin{table}[H]
   %\caption{ }
   \centering
   \begin{tabular}{| c | p{12 cm} |}
   \hline
    Place & GISTDA Satellite Operation Center, Geo-Informatics and Space Technology Development Agency, Sriracha, Chonburi, Thailand \\ \hline
    Date (Time) & June 1, 2017 (16:00--17:30) \\ \hline
    Counterparts & Mr. Wasanchai VONGSANTIVANICH, Satellite Systems Engineer \\ \hline   
   \end{tabular}
\end{table}

Main elements of the discussion:

\begin{itemize}

\item Presentation of our project.

\item Participation of GISTDA in regional data sharing cooperation, in particular through Sentinel Asia.

\item New developments in GISTDA.

\end{itemize}















