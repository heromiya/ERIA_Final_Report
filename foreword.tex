\chapter*{Foreword}

Today, space and geospatial technology (SGT) are no longer just fields of advanced technological development and scientific research, but they have become key component to help achieve sustainable development and strengthening resilience. They can improve the efficiency and resilience of industrial operations and effectively address issues in regional economic integration of  the ASEAN.

Based on this understanding, an ERIA study project "Applying Space-based Technology for building resilience in ASEAN region" was conducted in 2014 and concluded that geospatial technologies and space technologies have notable potentials to strengthen the resilience although sustainable mechanism for integrating the technologies  has not still been well established. The study pointed out necessity of (i) trans-border mechanism to deliver the geospatial and space-based information from data provider to end users in disaster-affected areas with support of international activities and (ii) financial schemes involving private sectors, or public-private partnerships (PPP), to operate collaborative integration of the technologies in sustainable and practical manners.

To implement such a mechanism, assessing the benefits from SGTs and applications and conceptualizing necessary policies are very important.
This report provides current status of SGT, applications, and potential benefits to the ASEAN based on the past practices in Asia and Pacific. It includes information about what technology and combinations of technologies were applied and how those contributes to resilience of ASEAN by key issues including urban development, infrastructure planning and management, transportation management, improving Quality of Life, post-disaster management, improving logistics efficiency, sustainable operations of agriculture and fishery, improving efficiency manufacturing and service industry, management of environmental services and natural resources.

Besides, This report covers policy perspectives with strategy options about how to implement the regional connectivity of ASEAN supported by the technologies, especially focusing on trans-border mechanism of data and information.

We hope that this report proves useful to the ASEAN, the participating states, and stakeholders who work with the ASEAN by guiding the SGT applications to the more connectivity of ASEAN states.

\begin{flushright}
\vspace{1 cm}
{\Large \bfseries Prof.~Ryosuke \textsc{Shibasaki}\par}
\vspace{0.2 cm}
{\itshape Professor, Center for Spatial Information Science\par}
{\itshape The University of Tokyo\par}
\end{flushright}
