\chapter{Summary of the workshops}


\section{ERIA Forum during APRSAF -- Manila, November 2016}

\subsection{Agenda}

{\flushleft

\textbf{Time}: 15:00-18:00, 18 November 2016

\vspace{0.4 cm}

\textbf{Venue}: Leyte Room, Sofitel Philippine Plaza Manila

\vspace{0.4 cm}

\textbf{Participants}:

\begin{itemize}
\item ASEAN -- 10 people from Indonesia, Malaysia, Philippines, Thailand, Vietnam
\item Japan -- 20 people incl. DDG of Cabinet Office, NIED, RESTEC, JSS, JSF, universities (UT, KIT, Yamaguchi), PASCO and AXELSPACE
\end{itemize}

\vspace{0.4 cm}

\textbf{Agenda}:

\begin{enumerate}

\item Opening and Overview of ERIA project

	\begin{enumerate}
	\item Opening Remarks (3M)
	
	\textit{Cabinet Office, GoJ}
	\item ASEAN Roadmap for Regional DRM Cooperation: Objective of ERIA Forum and Discussion Points (10M)
	
	\textit{Hiroyuki Miyazaki, The University of Tokyo}
	\item Perspectives of ERIA and ADB's Strategies in Asian Cooperation (5M)
	
	\textit{Quentin Verspieren, The University of Tokyo}
	\end{enumerate}

\item Toward Enhancement of Resilience by Using Space Technology

	\begin{enumerate}

	\item Geospatial Information (Positioning Information, Big Data, GNSS Reference Station, etc.)
	
		\begin{enumerate}
		\item Deployment of GNSS Reference Station (8M)
		
		\textit{Tadashi Sasagawa, PASCO}
		\item Expectation for High Precise Positioning Service (7M)
		
		\textit{Peraya Tantianuparp, Hydro and Agro Informatics Institute (HAII)}
		\item Case Studies of DRM services using Geospatial Information/Positioning Technologies (15M)
		
		\textit{Masahiko Nagai, Yamaguchi University}
		\item Remote Sensing and Big Data Analysis for DRM (5M)
		
		\textit{Hiroyuki Miyazaki, The University of Tokyo}
		\end{enumerate} 

	\item Utilization of Satellite Data
		
		\begin{enumerate}
		\item SDGs and Climate Change Adaptation—A Pilot Project on Index-based Insurance for Agriculture (10M)
		
		\textit{Kazushi Motomura, RESTEC}
		\item Ocean Monitoring (e.g. Oil Spill, Coastal Zone Monitoring): Utilization of Satellite Data (ASTER, Hyperspectral Data, etc.) (15M)
		
		\textit{I Nyoman Radiarta, Institute for Marine Research and Observation (IMRO)}
		
		\textit{Osamu Kashimura, Japan Space Systems}
		\item Store and Forward Communication (5M)
		
		\textit{Toshihiro Obata, The University of Tokyo}
		\item Nano-satellite (10M)
		
		\textit{George Maeda, Kyushu Institute of Technology}
		\item Small Satellite Constellation (7M)
		
		\textit{Yuya Nakamura, AXELSPACE}
		\item Introduction of DIAS and Himawari Data (10M)
		
		\textit{Hiroyuki Miyazaki, The University of Tokyo (on behalf of Prof. Ryosuke Shibasaki)}
		\end{enumerate}

	\end{enumerate}
	
\item Workshop (60M)
	
	\begin{itemize}
	\item Brainstorming of Core Services and Symbolic Projects for Future Cooperation
	\item Needs from ASEAN Countries
	\end{itemize} 

\item Closing Remarks
	
\textit{Cabinet Office, GoJ}

\end{enumerate}
}

\subsection{Details on the workshop}

\tab The objective of this workshop was to identify core-examples of the utilization of SGT in ASEAN, through intense discussions among participating specialists. Then a specific focus was given on the potential for ASEAN-wide scale up of the examples.

\vspace{0.4 cm}

The guest specialists were divided into four groups moderated by organizers associating a specific issue (disaster risk management or ocean-related applications) with a specific technology (Earth observation or position systems):

\begin{enumerate}

\item Disaster risk management \& Earth Observation

\item Disaster risk management \& Positioning

\item Ocean \& Earth Observation

\item Ocean \& Positioning

\end{enumerate}



\section{ERIA Wrap-up Forum -- Jakarta, July 2017} \label{ws_jakarta}


\subsection{Agenda}

{\flushleft

\textbf{Time}: 12:00-15:30, 6 July 2017 (11:00-12:00 Registration and Lunch)

\vspace{0.4 cm}

\textbf{Venue}: Headquarters of ERIA

\vspace{0.4 cm}

\textbf{Address}: Sentral Senayan II, 6th floor Jalan Asia Afrika No.8, Gelora Bung Karno, Senayan, Jakarta Pusat 10270, Indonesia

\vspace{0.4 cm}

\textbf{Participants}:

\begin{itemize}
\item ASEAN -- 10 people from Indonesia, Malaysia, Philippines, Thailand, Vietnam
\item Japan -- 20 people incl. DDG of Cabinet Office, NIED, RESTEC, JSS, JSF, universities (UT, KIT, Yamaguchi), PASCO and AXELSPACE
\end{itemize}

\vspace{0.4 cm}

\textbf{Agenda}:

\begin{enumerate}

\item Opening and Overview of ERIA project

	\begin{enumerate}
	\item Opening Remarks (3M)
	
	\textit{Mr. Takashi Ikeda, Cabinet Office, GoJ}
	\item ASEAN Roadmap for Regional DRM Cooperation: Objective of ERIA Forum and Discussion Points (20M and 10M Q\&A)
	
	\textit{Prof. Ryosuke Shibasaki, The University of Tokyo}
	
	\textit{Prof. Takayoshi Fukuyo, The University of Tokyo}
	
	\textit{Prof. Hiroyuki Miyazaki, The University of Tokyo}
	
	\textit{Mr. Quentin Verspieren, The University of Tokyo}
	\item Perspectives of ERIA and ADB's Strategies in Asian Cooperation (5M)
	
	\textit{Quentin Verspieren, The University of Tokyo}
	\end{enumerate}

\item Possible Core Examples Toward Envisaged Sustainable Scheme of Geospatial/ Space‐based Service in ASEAN Region

	\begin{enumerate}
	\item Indonesia
		
	\textit{Dr. I Nyoman Radiarta, IMRO (10M and 3M Q\&A)}
	
	\textit{Mr. M. Rokhis Khomarudin, Remote Sensing Application Center, LAPAN}
	\item Lao PDR (12M and 3M Q\&A)
		
	\textit{Ms. Xaysomphone Souvannavong, Department of Disaster Management and Climate Change, Ministry of Natural Resource and Environment}
	\item Malaysia (12M and 3M Q\&A)
		
	\textit{Dr. Abdul Rashid Mohamed Shariff, Universiti Putra Malaysia (UPM)}
	\item Myanmar (12M and 3M Q\&A)
		
	\textit{Ms. Daw Thiri Maung, Ministry of Social Welfare, Relief and Resettlement}
	\item Philippines (12M and 3M Q\&A)
		
	\textit{Mr. Adrian Josele Quional/ Mr. Alvin Familara, National SPACE Development Program}
	\item Thailand (12M and 3M Q\&A)
		
	\textit{Mr.Jakrapong Tawala, GISTDA}
	\item Vietnam (12M and 3M Q\&A)
		
	\textit{Mr. Vu Anh Tuan, Vietnam National Satellite Center (VNSC)/Vietnam Academy of Science and Technology (VAST)}
	\end{enumerate} 

\item Discussions (60M)
	
	\begin{itemize}
	\item Developing sustainable collaboration model for implementing integrated space‐ based/geospatial socio-economic platform to strengthen the resilience in ASEAN community
	\item Next Step of the Project
	\end{itemize} 

\item Closing Remarks
	
\textit{Cabinet Office, GoJ}

\end{enumerate}
}

\subsection{Details on the workshop}

The discussion part witnessed very interesting debates and comments from all present members, which spirit can be found in the 'abstract' they send us after the workshop. These abstracts can be found, unedited, in appendix \ref{abstracts}.








